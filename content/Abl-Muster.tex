% !TEX root = ../AGFA-Abl-Master.tex
% -------------------------- 
%%  Thema:  Muster für Aufgabenblatt
%%  Stand:	2022/10/19      
%% --------------------------

%% -- Nummer des Blattes
%% --
\setcounter{nummer}{1}

%% -- Spruch
%% --
\dictum[Klaus Havenstein]{Am schönsten ist es, nichts zu tun und dann vom Nichtstun auszuruh’n.}

%% -- Falls im Spaltemodus erstellt werden soll
%% --
\begin{multicols}{2}[%
\section*{Aufgabenblatt \thenummer}%
Kurzer Hinweis, dass diese Aufgaben schriftlich bis wann zu bearbeiten sind oder andere Informationen für die Teilnehmer der Übungsgruppe.
Ab hier kann man dann auch zweispaltig weitermachen.]
\end{multicols}		% wenn zweispaltig -> auskommentieren
%% --
\RaggedRight			% Besser Flattersatz machen
%% --
%% --
\begin{exercise}\label{exerc:exponential}
Für eine Matrix $ A \in \Mat( n , n ; \C) $ definieren wir
%
\[
	\e^{ t A } := \sum_{ k=0 }^{ \infty } \frac{ t^{ k } A^{ k }}{ k! } .
\]
%
Zeige, dass die Abbildung
%
\[
	t \mapsto \e^{ tA } \colon \R \to \Mat( n , n ; \C)
\]
%
stetig differenzierbar mit Ableitung
%
\[
	\frac{ \mathrm{d}}{\mathrm{d}t} \e^{ t A } = A \cdot \e^{ t A }
\]
%
ist.
\end{exercise}
%%
\begin{exercise}\label{exerc:berechnen}
Berechne $ \e^{ tA } $ für $ \lambda \in \C $. 
%%
\begin{enumerate}[(i)]

	\item
	\(				
		A = 
		\begin{pmatrix}
			0	& 	1 \\
			1	&	0 \\
		\end{pmatrix}.
	\)
	%
	\item
	
	\(
	A =
	\begin{pNiceMatrix} % ins Manual von nicematrix sehen; Seite 25ff
	\lambda 	& 1 			& 0		& \Cdots  	& \Cdots  	&	 0		\\
	0 		& \lambda 	& 1 	& 0			& \Cdots   	& 0			\\
	\Vdots  	&  			& \Ddots 	& \Ddots 	&		& \Vdots    \\
	\\ 
	&&&&&1\\
	0	& \Cdots 	&  	&   &  							& \lambda	\\
	\end{pNiceMatrix} 
	\)
\end{enumerate}
\end{exercise}
%%
\begin{solution}[zu Aufgabe~\ref{exerc:exponential}]

\blindtext
\end{solution}

%% -- Ende Text
%\end{multicols}		% wenn zweispaltig: Kommentar % wegmachen

%% ---------------------------------------------
%% -- Abl-layout.tex
%% -- Vorlage für Aufgabenblätter in der AGFA
%% Stand: 2022/10/19
%% ---------------------------------------------

%%-- Deutsch mit englischer Unterstützung
%%-- 
\usepackage[english,main=ngerman]{babel}
	\babelprovide[hyphenrules=ngerman-x-latest]{ngerman}
\usepackage[autostyle,german=guillemets,english=american]{csquotes}

%%-- Kopf- und Fußzeile
%%
\usepackage[automark]{scrlayer-scrpage}

\newcommand{\seitennummer}{%
	\makebox[0pt][l]{%
    \makebox[\marginparsep][r]{%
      \rule[-\dp\strutbox]{1pt}{\normalbaselineskip}\nobreakspace
    }%
    \makebox[\marginparwidth][l]{\pagemark}}
    }
    
\ohead{\seitennummer}%
\ofoot{}
%
\chead{Vorlesung Kontrolltheorie} 
\cfoot{}
%
\ihead{R. Nagel} 
\ifoot{}
\KOMAoptions{headsepline=0.8pt}
\pagestyle{scrheadings}	

%% --  Ausgabe mit Hinweis auf Lochung
%% --
\usepackage{eso-pic}
\AddToShipoutPicture{%
  \AtPageLowerLeft{%
 %   \put(\LenToUnit{8mm},\LenToUnit{108mm}){\circle*{\LenToUnit{5mm}}}%
  %  \put(\LenToUnit{8mm},\LenToUnit{188mm}){\circle*{\LenToUnit{5mm}}}%
    \put(\LenToUnit{5mm},\LenToUnit{148.5mm}){\line(1,0){\LenToUnit{5mm}}}%
  }%
}

%%-- Nummer des Arbeitsblattes: Bitte Nummer im Include-File setzen
%%--
\newcounter{nummer}

%%-- Pakete
%%--
\usepackage{%
	,multicol		% Wenn zweispaltig
	,ragged2e		% Flattersatz machen
	,libertine		% Schriftsatz
	}

\usepackage[inline,shortlabels]{enumitem}	

\usepackage{nicematrix}		% texdoc nicematrix
\NiceMatrixOptions{cell-space-limits = 1pt}


\usepackage{%		% AMSmath
	,amsmath
	,amssymb}
\usepackage[tbtags]{mathtools}
\usepackage{ntheorem}

%% -- Umgebung Aufgabe und Lösung
%% --
\theorembodyfont{\normalfont}
\newtheorem{exercise}{Aufgabe}
\newtheorem{aufgabe}[exercise]{Aufgabe}
\newtheorem*{solution}{Lösungsvorschlag}
\newtheorem*{loesung}{Lösungsvorschlag}